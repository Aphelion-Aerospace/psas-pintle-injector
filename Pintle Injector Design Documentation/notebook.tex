
% Default to the notebook output style

    


% Inherit from the specified cell style.




    
\documentclass[11pt]{article}

    
    
    \usepackage[T1]{fontenc}
    % Nicer default font (+ math font) than Computer Modern for most use cases
    \usepackage{mathpazo}

    % Basic figure setup, for now with no caption control since it's done
    % automatically by Pandoc (which extracts ![](path) syntax from Markdown).
    \usepackage{graphicx}
    % We will generate all images so they have a width \maxwidth. This means
    % that they will get their normal width if they fit onto the page, but
    % are scaled down if they would overflow the margins.
    \makeatletter
    \def\maxwidth{\ifdim\Gin@nat@width>\linewidth\linewidth
    \else\Gin@nat@width\fi}
    \makeatother
    \let\Oldincludegraphics\includegraphics
    % Set max figure width to be 80% of text width, for now hardcoded.
    \renewcommand{\includegraphics}[1]{\Oldincludegraphics[width=.8\maxwidth]{#1}}
    % Ensure that by default, figures have no caption (until we provide a
    % proper Figure object with a Caption API and a way to capture that
    % in the conversion process - todo).
    \usepackage{caption}
    \DeclareCaptionLabelFormat{nolabel}{}
    \captionsetup{labelformat=nolabel}

    \usepackage{adjustbox} % Used to constrain images to a maximum size 
    \usepackage{xcolor} % Allow colors to be defined
    \usepackage{enumerate} % Needed for markdown enumerations to work
    \usepackage{geometry} % Used to adjust the document margins
    \usepackage{amsmath} % Equations
    \usepackage{amssymb} % Equations
    \usepackage{textcomp} % defines textquotesingle
    % Hack from http://tex.stackexchange.com/a/47451/13684:
    \AtBeginDocument{%
        \def\PYZsq{\textquotesingle}% Upright quotes in Pygmentized code
    }
    \usepackage{upquote} % Upright quotes for verbatim code
    \usepackage{eurosym} % defines \euro
    \usepackage[mathletters]{ucs} % Extended unicode (utf-8) support
    \usepackage[utf8x]{inputenc} % Allow utf-8 characters in the tex document
    \usepackage{fancyvrb} % verbatim replacement that allows latex
    \usepackage{grffile} % extends the file name processing of package graphics 
                         % to support a larger range 
    % The hyperref package gives us a pdf with properly built
    % internal navigation ('pdf bookmarks' for the table of contents,
    % internal cross-reference links, web links for URLs, etc.)
    \usepackage{hyperref}
    \usepackage{longtable} % longtable support required by pandoc >1.10
    \usepackage{booktabs}  % table support for pandoc > 1.12.2
    \usepackage[inline]{enumitem} % IRkernel/repr support (it uses the enumerate* environment)
    \usepackage[normalem]{ulem} % ulem is needed to support strikethroughs (\sout)
                                % normalem makes italics be italics, not underlines
    

    
    
    % Colors for the hyperref package
    \definecolor{urlcolor}{rgb}{0,.145,.698}
    \definecolor{linkcolor}{rgb}{.71,0.21,0.01}
    \definecolor{citecolor}{rgb}{.12,.54,.11}

    % ANSI colors
    \definecolor{ansi-black}{HTML}{3E424D}
    \definecolor{ansi-black-intense}{HTML}{282C36}
    \definecolor{ansi-red}{HTML}{E75C58}
    \definecolor{ansi-red-intense}{HTML}{B22B31}
    \definecolor{ansi-green}{HTML}{00A250}
    \definecolor{ansi-green-intense}{HTML}{007427}
    \definecolor{ansi-yellow}{HTML}{DDB62B}
    \definecolor{ansi-yellow-intense}{HTML}{B27D12}
    \definecolor{ansi-blue}{HTML}{208FFB}
    \definecolor{ansi-blue-intense}{HTML}{0065CA}
    \definecolor{ansi-magenta}{HTML}{D160C4}
    \definecolor{ansi-magenta-intense}{HTML}{A03196}
    \definecolor{ansi-cyan}{HTML}{60C6C8}
    \definecolor{ansi-cyan-intense}{HTML}{258F8F}
    \definecolor{ansi-white}{HTML}{C5C1B4}
    \definecolor{ansi-white-intense}{HTML}{A1A6B2}

    % commands and environments needed by pandoc snippets
    % extracted from the output of `pandoc -s`
    \providecommand{\tightlist}{%
      \setlength{\itemsep}{0pt}\setlength{\parskip}{0pt}}
    \DefineVerbatimEnvironment{Highlighting}{Verbatim}{commandchars=\\\{\}}
    % Add ',fontsize=\small' for more characters per line
    \newenvironment{Shaded}{}{}
    \newcommand{\KeywordTok}[1]{\textcolor[rgb]{0.00,0.44,0.13}{\textbf{{#1}}}}
    \newcommand{\DataTypeTok}[1]{\textcolor[rgb]{0.56,0.13,0.00}{{#1}}}
    \newcommand{\DecValTok}[1]{\textcolor[rgb]{0.25,0.63,0.44}{{#1}}}
    \newcommand{\BaseNTok}[1]{\textcolor[rgb]{0.25,0.63,0.44}{{#1}}}
    \newcommand{\FloatTok}[1]{\textcolor[rgb]{0.25,0.63,0.44}{{#1}}}
    \newcommand{\CharTok}[1]{\textcolor[rgb]{0.25,0.44,0.63}{{#1}}}
    \newcommand{\StringTok}[1]{\textcolor[rgb]{0.25,0.44,0.63}{{#1}}}
    \newcommand{\CommentTok}[1]{\textcolor[rgb]{0.38,0.63,0.69}{\textit{{#1}}}}
    \newcommand{\OtherTok}[1]{\textcolor[rgb]{0.00,0.44,0.13}{{#1}}}
    \newcommand{\AlertTok}[1]{\textcolor[rgb]{1.00,0.00,0.00}{\textbf{{#1}}}}
    \newcommand{\FunctionTok}[1]{\textcolor[rgb]{0.02,0.16,0.49}{{#1}}}
    \newcommand{\RegionMarkerTok}[1]{{#1}}
    \newcommand{\ErrorTok}[1]{\textcolor[rgb]{1.00,0.00,0.00}{\textbf{{#1}}}}
    \newcommand{\NormalTok}[1]{{#1}}
    
    % Additional commands for more recent versions of Pandoc
    \newcommand{\ConstantTok}[1]{\textcolor[rgb]{0.53,0.00,0.00}{{#1}}}
    \newcommand{\SpecialCharTok}[1]{\textcolor[rgb]{0.25,0.44,0.63}{{#1}}}
    \newcommand{\VerbatimStringTok}[1]{\textcolor[rgb]{0.25,0.44,0.63}{{#1}}}
    \newcommand{\SpecialStringTok}[1]{\textcolor[rgb]{0.73,0.40,0.53}{{#1}}}
    \newcommand{\ImportTok}[1]{{#1}}
    \newcommand{\DocumentationTok}[1]{\textcolor[rgb]{0.73,0.13,0.13}{\textit{{#1}}}}
    \newcommand{\AnnotationTok}[1]{\textcolor[rgb]{0.38,0.63,0.69}{\textbf{\textit{{#1}}}}}
    \newcommand{\CommentVarTok}[1]{\textcolor[rgb]{0.38,0.63,0.69}{\textbf{\textit{{#1}}}}}
    \newcommand{\VariableTok}[1]{\textcolor[rgb]{0.10,0.09,0.49}{{#1}}}
    \newcommand{\ControlFlowTok}[1]{\textcolor[rgb]{0.00,0.44,0.13}{\textbf{{#1}}}}
    \newcommand{\OperatorTok}[1]{\textcolor[rgb]{0.40,0.40,0.40}{{#1}}}
    \newcommand{\BuiltInTok}[1]{{#1}}
    \newcommand{\ExtensionTok}[1]{{#1}}
    \newcommand{\PreprocessorTok}[1]{\textcolor[rgb]{0.74,0.48,0.00}{{#1}}}
    \newcommand{\AttributeTok}[1]{\textcolor[rgb]{0.49,0.56,0.16}{{#1}}}
    \newcommand{\InformationTok}[1]{\textcolor[rgb]{0.38,0.63,0.69}{\textbf{\textit{{#1}}}}}
    \newcommand{\WarningTok}[1]{\textcolor[rgb]{0.38,0.63,0.69}{\textbf{\textit{{#1}}}}}
    
    
    % Define a nice break command that doesn't care if a line doesn't already
    % exist.
    \def\br{\hspace*{\fill} \\* }
    % Math Jax compatability definitions
    \def\gt{>}
    \def\lt{<}
    % Document parameters
    \title{00 - Pintle Injector Theory, Design, and Testing}
    
    
    

    % Pygments definitions
    
\makeatletter
\def\PY@reset{\let\PY@it=\relax \let\PY@bf=\relax%
    \let\PY@ul=\relax \let\PY@tc=\relax%
    \let\PY@bc=\relax \let\PY@ff=\relax}
\def\PY@tok#1{\csname PY@tok@#1\endcsname}
\def\PY@toks#1+{\ifx\relax#1\empty\else%
    \PY@tok{#1}\expandafter\PY@toks\fi}
\def\PY@do#1{\PY@bc{\PY@tc{\PY@ul{%
    \PY@it{\PY@bf{\PY@ff{#1}}}}}}}
\def\PY#1#2{\PY@reset\PY@toks#1+\relax+\PY@do{#2}}

\expandafter\def\csname PY@tok@w\endcsname{\def\PY@tc##1{\textcolor[rgb]{0.73,0.73,0.73}{##1}}}
\expandafter\def\csname PY@tok@c\endcsname{\let\PY@it=\textit\def\PY@tc##1{\textcolor[rgb]{0.25,0.50,0.50}{##1}}}
\expandafter\def\csname PY@tok@cp\endcsname{\def\PY@tc##1{\textcolor[rgb]{0.74,0.48,0.00}{##1}}}
\expandafter\def\csname PY@tok@k\endcsname{\let\PY@bf=\textbf\def\PY@tc##1{\textcolor[rgb]{0.00,0.50,0.00}{##1}}}
\expandafter\def\csname PY@tok@kp\endcsname{\def\PY@tc##1{\textcolor[rgb]{0.00,0.50,0.00}{##1}}}
\expandafter\def\csname PY@tok@kt\endcsname{\def\PY@tc##1{\textcolor[rgb]{0.69,0.00,0.25}{##1}}}
\expandafter\def\csname PY@tok@o\endcsname{\def\PY@tc##1{\textcolor[rgb]{0.40,0.40,0.40}{##1}}}
\expandafter\def\csname PY@tok@ow\endcsname{\let\PY@bf=\textbf\def\PY@tc##1{\textcolor[rgb]{0.67,0.13,1.00}{##1}}}
\expandafter\def\csname PY@tok@nb\endcsname{\def\PY@tc##1{\textcolor[rgb]{0.00,0.50,0.00}{##1}}}
\expandafter\def\csname PY@tok@nf\endcsname{\def\PY@tc##1{\textcolor[rgb]{0.00,0.00,1.00}{##1}}}
\expandafter\def\csname PY@tok@nc\endcsname{\let\PY@bf=\textbf\def\PY@tc##1{\textcolor[rgb]{0.00,0.00,1.00}{##1}}}
\expandafter\def\csname PY@tok@nn\endcsname{\let\PY@bf=\textbf\def\PY@tc##1{\textcolor[rgb]{0.00,0.00,1.00}{##1}}}
\expandafter\def\csname PY@tok@ne\endcsname{\let\PY@bf=\textbf\def\PY@tc##1{\textcolor[rgb]{0.82,0.25,0.23}{##1}}}
\expandafter\def\csname PY@tok@nv\endcsname{\def\PY@tc##1{\textcolor[rgb]{0.10,0.09,0.49}{##1}}}
\expandafter\def\csname PY@tok@no\endcsname{\def\PY@tc##1{\textcolor[rgb]{0.53,0.00,0.00}{##1}}}
\expandafter\def\csname PY@tok@nl\endcsname{\def\PY@tc##1{\textcolor[rgb]{0.63,0.63,0.00}{##1}}}
\expandafter\def\csname PY@tok@ni\endcsname{\let\PY@bf=\textbf\def\PY@tc##1{\textcolor[rgb]{0.60,0.60,0.60}{##1}}}
\expandafter\def\csname PY@tok@na\endcsname{\def\PY@tc##1{\textcolor[rgb]{0.49,0.56,0.16}{##1}}}
\expandafter\def\csname PY@tok@nt\endcsname{\let\PY@bf=\textbf\def\PY@tc##1{\textcolor[rgb]{0.00,0.50,0.00}{##1}}}
\expandafter\def\csname PY@tok@nd\endcsname{\def\PY@tc##1{\textcolor[rgb]{0.67,0.13,1.00}{##1}}}
\expandafter\def\csname PY@tok@s\endcsname{\def\PY@tc##1{\textcolor[rgb]{0.73,0.13,0.13}{##1}}}
\expandafter\def\csname PY@tok@sd\endcsname{\let\PY@it=\textit\def\PY@tc##1{\textcolor[rgb]{0.73,0.13,0.13}{##1}}}
\expandafter\def\csname PY@tok@si\endcsname{\let\PY@bf=\textbf\def\PY@tc##1{\textcolor[rgb]{0.73,0.40,0.53}{##1}}}
\expandafter\def\csname PY@tok@se\endcsname{\let\PY@bf=\textbf\def\PY@tc##1{\textcolor[rgb]{0.73,0.40,0.13}{##1}}}
\expandafter\def\csname PY@tok@sr\endcsname{\def\PY@tc##1{\textcolor[rgb]{0.73,0.40,0.53}{##1}}}
\expandafter\def\csname PY@tok@ss\endcsname{\def\PY@tc##1{\textcolor[rgb]{0.10,0.09,0.49}{##1}}}
\expandafter\def\csname PY@tok@sx\endcsname{\def\PY@tc##1{\textcolor[rgb]{0.00,0.50,0.00}{##1}}}
\expandafter\def\csname PY@tok@m\endcsname{\def\PY@tc##1{\textcolor[rgb]{0.40,0.40,0.40}{##1}}}
\expandafter\def\csname PY@tok@gh\endcsname{\let\PY@bf=\textbf\def\PY@tc##1{\textcolor[rgb]{0.00,0.00,0.50}{##1}}}
\expandafter\def\csname PY@tok@gu\endcsname{\let\PY@bf=\textbf\def\PY@tc##1{\textcolor[rgb]{0.50,0.00,0.50}{##1}}}
\expandafter\def\csname PY@tok@gd\endcsname{\def\PY@tc##1{\textcolor[rgb]{0.63,0.00,0.00}{##1}}}
\expandafter\def\csname PY@tok@gi\endcsname{\def\PY@tc##1{\textcolor[rgb]{0.00,0.63,0.00}{##1}}}
\expandafter\def\csname PY@tok@gr\endcsname{\def\PY@tc##1{\textcolor[rgb]{1.00,0.00,0.00}{##1}}}
\expandafter\def\csname PY@tok@ge\endcsname{\let\PY@it=\textit}
\expandafter\def\csname PY@tok@gs\endcsname{\let\PY@bf=\textbf}
\expandafter\def\csname PY@tok@gp\endcsname{\let\PY@bf=\textbf\def\PY@tc##1{\textcolor[rgb]{0.00,0.00,0.50}{##1}}}
\expandafter\def\csname PY@tok@go\endcsname{\def\PY@tc##1{\textcolor[rgb]{0.53,0.53,0.53}{##1}}}
\expandafter\def\csname PY@tok@gt\endcsname{\def\PY@tc##1{\textcolor[rgb]{0.00,0.27,0.87}{##1}}}
\expandafter\def\csname PY@tok@err\endcsname{\def\PY@bc##1{\setlength{\fboxsep}{0pt}\fcolorbox[rgb]{1.00,0.00,0.00}{1,1,1}{\strut ##1}}}
\expandafter\def\csname PY@tok@kc\endcsname{\let\PY@bf=\textbf\def\PY@tc##1{\textcolor[rgb]{0.00,0.50,0.00}{##1}}}
\expandafter\def\csname PY@tok@kd\endcsname{\let\PY@bf=\textbf\def\PY@tc##1{\textcolor[rgb]{0.00,0.50,0.00}{##1}}}
\expandafter\def\csname PY@tok@kn\endcsname{\let\PY@bf=\textbf\def\PY@tc##1{\textcolor[rgb]{0.00,0.50,0.00}{##1}}}
\expandafter\def\csname PY@tok@kr\endcsname{\let\PY@bf=\textbf\def\PY@tc##1{\textcolor[rgb]{0.00,0.50,0.00}{##1}}}
\expandafter\def\csname PY@tok@bp\endcsname{\def\PY@tc##1{\textcolor[rgb]{0.00,0.50,0.00}{##1}}}
\expandafter\def\csname PY@tok@fm\endcsname{\def\PY@tc##1{\textcolor[rgb]{0.00,0.00,1.00}{##1}}}
\expandafter\def\csname PY@tok@vc\endcsname{\def\PY@tc##1{\textcolor[rgb]{0.10,0.09,0.49}{##1}}}
\expandafter\def\csname PY@tok@vg\endcsname{\def\PY@tc##1{\textcolor[rgb]{0.10,0.09,0.49}{##1}}}
\expandafter\def\csname PY@tok@vi\endcsname{\def\PY@tc##1{\textcolor[rgb]{0.10,0.09,0.49}{##1}}}
\expandafter\def\csname PY@tok@vm\endcsname{\def\PY@tc##1{\textcolor[rgb]{0.10,0.09,0.49}{##1}}}
\expandafter\def\csname PY@tok@sa\endcsname{\def\PY@tc##1{\textcolor[rgb]{0.73,0.13,0.13}{##1}}}
\expandafter\def\csname PY@tok@sb\endcsname{\def\PY@tc##1{\textcolor[rgb]{0.73,0.13,0.13}{##1}}}
\expandafter\def\csname PY@tok@sc\endcsname{\def\PY@tc##1{\textcolor[rgb]{0.73,0.13,0.13}{##1}}}
\expandafter\def\csname PY@tok@dl\endcsname{\def\PY@tc##1{\textcolor[rgb]{0.73,0.13,0.13}{##1}}}
\expandafter\def\csname PY@tok@s2\endcsname{\def\PY@tc##1{\textcolor[rgb]{0.73,0.13,0.13}{##1}}}
\expandafter\def\csname PY@tok@sh\endcsname{\def\PY@tc##1{\textcolor[rgb]{0.73,0.13,0.13}{##1}}}
\expandafter\def\csname PY@tok@s1\endcsname{\def\PY@tc##1{\textcolor[rgb]{0.73,0.13,0.13}{##1}}}
\expandafter\def\csname PY@tok@mb\endcsname{\def\PY@tc##1{\textcolor[rgb]{0.40,0.40,0.40}{##1}}}
\expandafter\def\csname PY@tok@mf\endcsname{\def\PY@tc##1{\textcolor[rgb]{0.40,0.40,0.40}{##1}}}
\expandafter\def\csname PY@tok@mh\endcsname{\def\PY@tc##1{\textcolor[rgb]{0.40,0.40,0.40}{##1}}}
\expandafter\def\csname PY@tok@mi\endcsname{\def\PY@tc##1{\textcolor[rgb]{0.40,0.40,0.40}{##1}}}
\expandafter\def\csname PY@tok@il\endcsname{\def\PY@tc##1{\textcolor[rgb]{0.40,0.40,0.40}{##1}}}
\expandafter\def\csname PY@tok@mo\endcsname{\def\PY@tc##1{\textcolor[rgb]{0.40,0.40,0.40}{##1}}}
\expandafter\def\csname PY@tok@ch\endcsname{\let\PY@it=\textit\def\PY@tc##1{\textcolor[rgb]{0.25,0.50,0.50}{##1}}}
\expandafter\def\csname PY@tok@cm\endcsname{\let\PY@it=\textit\def\PY@tc##1{\textcolor[rgb]{0.25,0.50,0.50}{##1}}}
\expandafter\def\csname PY@tok@cpf\endcsname{\let\PY@it=\textit\def\PY@tc##1{\textcolor[rgb]{0.25,0.50,0.50}{##1}}}
\expandafter\def\csname PY@tok@c1\endcsname{\let\PY@it=\textit\def\PY@tc##1{\textcolor[rgb]{0.25,0.50,0.50}{##1}}}
\expandafter\def\csname PY@tok@cs\endcsname{\let\PY@it=\textit\def\PY@tc##1{\textcolor[rgb]{0.25,0.50,0.50}{##1}}}

\def\PYZbs{\char`\\}
\def\PYZus{\char`\_}
\def\PYZob{\char`\{}
\def\PYZcb{\char`\}}
\def\PYZca{\char`\^}
\def\PYZam{\char`\&}
\def\PYZlt{\char`\<}
\def\PYZgt{\char`\>}
\def\PYZsh{\char`\#}
\def\PYZpc{\char`\%}
\def\PYZdl{\char`\$}
\def\PYZhy{\char`\-}
\def\PYZsq{\char`\'}
\def\PYZdq{\char`\"}
\def\PYZti{\char`\~}
% for compatibility with earlier versions
\def\PYZat{@}
\def\PYZlb{[}
\def\PYZrb{]}
\makeatother


    % Exact colors from NB
    \definecolor{incolor}{rgb}{0.0, 0.0, 0.5}
    \definecolor{outcolor}{rgb}{0.545, 0.0, 0.0}



    
    % Prevent overflowing lines due to hard-to-break entities
    \sloppy 
    % Setup hyperref package
    \hypersetup{
      breaklinks=true,  % so long urls are correctly broken across lines
      colorlinks=true,
      urlcolor=urlcolor,
      linkcolor=linkcolor,
      citecolor=citecolor,
      }
    % Slightly bigger margins than the latex defaults
    
    \geometry{verbose,tmargin=1in,bmargin=1in,lmargin=1in,rmargin=1in}
    
    

    \begin{document}
    
    
    \maketitle
    
    

    
    \section{Introduction}\label{introduction}

\begin{itemize}
\tightlist
\item
  When deciding to begin exploring liquid propellant rocket engines,
  PSAS decided to opt for a pintle type injector.
\item
  several reasons for this:

  \begin{itemize}
  \tightlist
  \item
    combustion stability
  \item
    throttling
  \item
    apparent ease of design
  \end{itemize}
\item
  Pintle injectors can be a good fit for small rocket motors, they have
  a number of challenges in both design and manufacture.
\end{itemize}

The Portland State Aerospace Society is in the process of testing its
\(2.2kN\) (\(500 lb_f\)) regeneratively cooled bi-propellant rocket
motor. One of the critical steps prior to hot firing the motor is the
testing and validating of the propellant injector. For this motor, a
pintle style injector was selected and an aluminum test article was
machined according to the designs provided by the 2015 Capstone team.

\subsection{PSAS's first bi-prop engine
attempt}\label{psass-first-bi-prop-engine-attempt}

The 2015 engine was designed to employ a liquid oxygen (LOX) centered
pintle design. In this design, LOX flows through the center of the
pintle and is sprayed outward in a radial pattern. The fuel enters a
chamber surrounding the base of the pintle and enters the engine through
the annular gap between the outer radius of the pintle and the engine.
(See Figures 1 \& 2) Inside of the combustion chamber, the two
propellants impinge upon each other at right angles. This, ideally,
results in a cone shaped spray of well mixed fuel and oxidizer.

\begin{quote}
Figure 1: Cross section of 2.2 kN engine. Fuel flow path shown in red,
LOX flow path shown in green.
\end{quote}

\begin{quote}
Figure 2: cross section view of genaric pintle V1 \& V2 geometry.
\end{quote}

    \section{Background}\label{background}

\subsection{Relevant history of the
pintle}\label{relevant-history-of-the-pintle}

\begin{itemize}
\tightlist
\item
  used in

  \begin{itemize}
  \item
    lunar decent module
  \item
    SpaceX Merlin Engine
  \item
  \end{itemize}
\item
  Not much published information with specifics on design, but some
  clues about the important parameters involved.
\end{itemize}

    \section{Theory}\label{theory}

\subsection{Operational Theory}\label{operational-theory}

The pintle injector serves two primary purposes in the engine. First, it
delivers the fuel and the oxidizer to the combustion chamber at the
correct mass flow rate and pressure. Secondly it mixes the propellants
and promotes proper combustion.

In practice, the mass flow rate through the injector is controlled by
the input pressure for the fuel and oxidizer upstream of the injector.
Ideally, the injector will have a pressure drop of between 15 and 25 \%
of the engine's chamber pressure.

\begin{quote}
This drop isolates chamber-pressure oscilations from the feed system,
reducing pressure coupling between the combustion chamber and the feed
system which could lead to instabilities or oscillations in the flow
that are driven by various in combustion. {[}1{]}
\end{quote}

For pintle injectors, this pressure drop can potentially go as low as
5\% {[}1{]}. Until PSAS has built some heritage with our pintle designs,
it is recommended that the 15-25\% range be aimed for.

According to some of the available research on pintle injectors, the
optimum propellant spray angle is \(45^\circ\). {[}2{]}

\begin{quote}
Figure 3: Side view of spray angle test for V1 pintle.
\end{quote}

\subsection{Key design Equations}\label{key-design-equations}

\subsubsection{Pressure loss
coefficient}\label{pressure-loss-coefficient}

The pressure loss coefficient is a key parameter in the design of
injectors. With the pressure loss coefficient known, it is possible to
determine the mass flow rate of each propellant through the injector by
measuring the pressure directly upstream of the injector and in the
chamber.

The governing equations employed for the first approximations in this
design were:

Mass flow rate:

\begin{equation}
 \dot{m} = \rho VA
\label{eq:massflow}
\tag{1}
\end{equation}

Where;

\(\dot{m} =\) Mass Flow Rate (from flow meter)

\$\rho = \$ Density of working fluid

\$V = \$ Mean fluid velocity

\$A = \$ Flow area

Given that high Reynolds numbers are expected, it is being assumed that
the pressure losses will be dominated by the inertial effects, rather
than viscous effects.

With this assumption, the energy equation can have the elevation,
velocity, and head loss due to friction neglected, which leaves:

\begin{equation}
P_1-P_2 = K_{L}\frac{\rho V_2^2}{2}
\label{eq:energy}
\tag{2}
\end{equation}

Where;

\$K\_L = \$ Pressure Loss coefficient\\
\(P_1-P_2 =\) Pressure drop

Equation 1 and 2 can be combined by substituting
\(V = \frac{\dot{m}}{\rho A}\) and rearranging to give;

\begin{equation}
\dot{m} = AC_d \sqrt{2\rho(\Delta P)}
\label{eq:massflow+cd}
\tag{3}
\end{equation}

which can be rewritten as;

\begin{equation}
C_d = \frac{\dot{m}}{A \sqrt{2\rho(\Delta P)}}
\label{eq:Cd}
\tag{4}
\end{equation}

Where;

\$C\_D = \$ Coefficient of discharge for Pintle and Annulus

\(A =\) Flow area(s) of Pintle or Annulus

\(\dot{m} =\) Mass Flow Rate (from flow meter)

\(\Delta P =\) Pressure upstream of Pintle or Annulus (gauge) -
atmospheric pressure (0 psi)

\$\rho = \$ Density of water (calculate based on temperature)

For the purposes of this loss coefficient, the area being used will be
the exit area of the injector.

Many references, but not all, use \(C_D\) for the loss coefficient for
injectors as shown in equation \ref{eq:Cd}. As a matter of preference,
the loss coefficient equations used within the jupyter notebook codes
developed on this project have used the following form of the loss
coefficient equations:

\begin{equation}
K_L = \frac{2\Delta P}{\rho \big(\frac{\dot{m}}{\rho A}\big)^2}
\label{eq:Kl}
\tag{7}
\end{equation}

or

\begin{equation}
K_L = \frac{2\Delta P}{\rho V^2}
\label{eq:Kl alt}
\tag{8}
\end{equation}

where; \$ K\_L = \frac{1}{\sqrt{C_d}}\$

Equation \ref{eq:Kl} shows that in order to empirically determine
\(K_L\), one must know the exit area(s) of the injector, the mass flow
rate and the corresponding pressure drop.

The exit orifices can be precisely measured using gauge pins, and the
mass flow rate and pressure drop can be measured in controlled
experiments.

\subsubsection{Momentum Ratio}\label{momentum-ratio}

According to available research, the spray angle of the mixed
propellants has an effect on engine performance{[}2{]}. Optimum
combustion is supposed to occur when the spray angle is approximately
\(45^\circ\). Theoretically this happens when the ratio of the momentum
rates of the two propellants is equal to 1.

This ratio can be expressed as;

\begin{equation}
\frac{\dot{m_f}V_f}{\dot{m_o}V_o} = 1
\label{eq:momentum}
\tag{9}
\end{equation}

By substituting the velocity in the momentum ratio for the mass flow
rate as defined in equation \ref{eq:massflow}, the optimum momentum
ratio equation can be rewritten as;

\begin{equation}
\frac{\dot{m_f}^2}{\rho_f A_f} = \frac{\dot{m_o}^2}{\rho_o A_o}
\label{eq:momentum_alt}
\tag{10}
\end{equation}

The calculations for the momentum ratio are slightly non-intuitive
because the area used in the calculation is not always the entire exit
area.

For the V1, V2, and V3 pintle designs, the exit areas used to calculate
for the momentum ratios were:

Pintle: \(\pi \frac{d_p^2}{4}\)

Annulus: \(d_p\times d_g\)

Where; \(d_p\) is equal to the diameter of the exit orifices of the
pintle, and \(d_g\) is equal to the annular gap defined as:

\$d\_g = \frac{1}{2}(A\_\{OD\} - P\_\{OD\}) \$

Where \(A_{OD}\) is the outer diameter of the Annulus, and \(P_{OD}\) is
the outer diameter of the pintle.

\begin{itemize}
\tightlist
\item
  Should probably include a drawing or diagram showing the area's
  involved
\end{itemize}

\subsubsection{Weber Number}\label{weber-number}

The Weber number is the ratio between inertial forces and surface
tension.

\begin{equation}
\frac{\rho V^2 D}{\sigma}
\label{eq:Weber}
\tag{12}
\end{equation}

Where; \(L\) is a characteristic length (exit diameter of holes, or
annular gap) \(\sigma\) is the surface tension

The weber number is an important parameter in the formation of droplets.
When a fluid jet has a high weber number at its exit, the inertial
forces cause the fluid to break up into small droplets. This is the
basic principal behind spray bottle atomizers.

It is suggested that the exit velocity of LOX remain above 10 m/s in
order to obtain a sufficiently high Weber number for acceptable droplet
formation. {[}3{]} The fuel should seek to have a weber number at least
as high as that of the oxidizer.

\subsubsection{Reynolds Number}\label{reynolds-number}

\begin{itemize}
\tightlist
\item
  justifies use of turbulent modeling for CFD
\end{itemize}

\begin{equation}
\frac{\rho V D}{\mu}
\label{eq:Reynolds}
\tag{11}
\end{equation}

Where; \(D\) is a characteristic length (exit diameter of holes, or the
hydraulic diameter of the annulus) \(\sigma\) is the surface tension.

    \section{Design}\label{design}

\subsection{Reality isn't always
convenient}\label{reality-isnt-always-convenient}

\subsubsection{Loss coefficients}\label{loss-coefficients}

Determining the loss coefficients for the a pintle style injector was
one of the first challenges to be overcome in designing an injector
which met both the mass flow rate and pressure requirements of PSAS's
first rocket engine.

The initial strategy was to break the internal flow paths of the
injector into sections and estimate loss coefficients for each section.

\begin{quote}
Figure 4: Cross section view of V1 pintle with locations of initial loss
coefficients used for LOX flow path
\end{quote}

\begin{quote}
Figure 5: Cross section view of V1 pintle with locations of initial loss
coefficients used for fuel flow path.
\end{quote}

The idea was that each flow path within the injector has multiple
changes to flow area between point of measurement and the atmosphere.
The early hand calculations in this project attempted to approximated
the overall loss coefficient by segregating the internal flow geometry
into segments which could examined individually for pressure losses.

The sum of those pressure losses could then be added together to
determine a total pressure drop estimate, \(\Delta P\). This
\(\Delta P\) could then be used along with the mass flow rate to
estimate a total \(K_L\) for the injector. This showed early promise
with the V2 injector and gave pressure drops which agreed with CFD
models and physical testing. When attempts were made to scale up the
injector design to meet the general requirements of an engine capable of
reaching 100km, the errors in the estimate quickly diverged from CFD
models.

\begin{itemize}
\item
  A note of caution, hand calculations and CFD should be employed during
  the design phase, but any injector which is to be used in an engine
  needs be specifically tested to determine its loss coefficients and
  spray angle performance. Small variations in hole sizes, even when
  within tolerance, can effect the actual performance of an injector.
  Theoretical models are valuable to steer the design into the right
  neighborhood, but still require physical testing for validation.
\item
  Loss coefficients for this design don't scale linearly
\end{itemize}

\subsubsection{Exit Velocity}\label{exit-velocity}

\begin{itemize}
\tightlist
\item
  Not always \(v = \frac{\dot{m}}{\rho A}\)
\item
  flow separation in the tip of the injector can reduce the actual flow
  area of the orifice by a significant amount. This increases the
  velocity and messes with the momentum ratio calculations.
\end{itemize}

\begin{quote}
Figure 6: CFD image showing flow velocity in a cross section of scaled
up V3 pintle tip. Mass flow rate was set to 1.5 kg/s.
\end{quote}

\subsection{Current Designs}\label{current-designs}

\subsubsection{Known issues}\label{known-issues}

\subsection{Future Designs}\label{future-designs}

\subsubsection{Ways to address known
issues}\label{ways-to-address-known-issues}

    \section{Manufacturing}\label{manufacturing}

\subsection{Test articles}\label{test-articles}

\subsubsection{Machined Aluminum}\label{machined-aluminum}

\begin{itemize}
\tightlist
\item
  Lots of work to do, requires careful machining and precision.
\end{itemize}

\subsubsection{3D printed}\label{d-printed}

\paragraph{MJP from 3D systems}\label{mjp-from-3d-systems}

This is by far the preferred option. Multiple prints were done and
tested using this printing method. The prints have very high dimensional
accuracy and can have NPT pipe threads printed into them directly
without the need to separately tap them. The same goes for helicoil
taps.

\paragraph{Other options}\label{other-options}

\begin{itemize}
\item
  FFF Don't bother
\item
  Formlabs - have not had success attempting to use a print from the
  Formlabs in the EPL. The material was too brittle and broke when
  attempting to tap threads into it. Could work if you oversize the
  entrance holes and epoxy in threaded hardware. The prints have some
  dimensional issues to pay attention to. Specifically the circularity
  of the pintle and the concentricity of the annulus.
\item
\end{itemize}

\subsection{Final injector}\label{final-injector}

Machine out of 304 or 316 SS. Probably best to have professionally done

    \section{Hydro Testing}\label{hydro-testing}

\subsection{Purpose and Theory}\label{purpose-and-theory}

\subsection{Testing equipment}\label{testing-equipment}

\subsection{Testing Procedures}\label{testing-procedures}

\subsection{Best Practices and
Recommendations}\label{best-practices-and-recommendations}

    \section{Cold Flow Testing}\label{cold-flow-testing}

\subsection{LN2 and water (or IPA)}\label{ln2-and-water-or-ipa}

\subsubsection{Some points to consider}\label{some-points-to-consider}

\begin{itemize}
\tightlist
\item
  Hasn't been done by PSAS at this time
\item
  LN2 is cold AF
\item
  Pressurized cryogenic gas can be dangerous even if the gas itself is
  inert
\item
  Need location and proper PPE
\end{itemize}

    \section{Hot Fire Testing}\label{hot-fire-testing}

    \section{References}\label{references}

{[}1{]} Humble, R. W., Henry, G. N., Larson, W. J. Space Propulsion
Analysis and Design, The McGraw-Hill Companies, Inc, New York, 1995

{[}2{]} Blakely, J., Freeberg, J., Hogge, J., Spray Cone Formation from
Pintle-Type Injector Systems in Liquid Rocket Engines, Student Paper,
2017

{[}3{]} Escher, Daric William. ``Design and Preliminary Hot Fire and
Cold Flow Testing of Pintle Injectors.'' Pennsylvania State University,
1996.


    % Add a bibliography block to the postdoc
    
    
    
    \end{document}
