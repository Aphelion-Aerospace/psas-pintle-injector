
% Default to the notebook output style

    


% Inherit from the specified cell style.




    
\documentclass[11pt]{article}

    
    
    \usepackage[T1]{fontenc}
    % Nicer default font (+ math font) than Computer Modern for most use cases
    \usepackage{mathpazo}

    % Basic figure setup, for now with no caption control since it's done
    % automatically by Pandoc (which extracts ![](path) syntax from Markdown).
    \usepackage{graphicx}
    % We will generate all images so they have a width \maxwidth. This means
    % that they will get their normal width if they fit onto the page, but
    % are scaled down if they would overflow the margins.
    \makeatletter
    \def\maxwidth{\ifdim\Gin@nat@width>\linewidth\linewidth
    \else\Gin@nat@width\fi}
    \makeatother
    \let\Oldincludegraphics\includegraphics
    % Set max figure width to be 80% of text width, for now hardcoded.
    \renewcommand{\includegraphics}[1]{\Oldincludegraphics[width=.8\maxwidth]{#1}}
    % Ensure that by default, figures have no caption (until we provide a
    % proper Figure object with a Caption API and a way to capture that
    % in the conversion process - todo).
    \usepackage{caption}
    \DeclareCaptionLabelFormat{nolabel}{}
    \captionsetup{labelformat=nolabel}

    \usepackage{adjustbox} % Used to constrain images to a maximum size 
    \usepackage{xcolor} % Allow colors to be defined
    \usepackage{enumerate} % Needed for markdown enumerations to work
    \usepackage{geometry} % Used to adjust the document margins
    \usepackage{amsmath} % Equations
    \usepackage{amssymb} % Equations
    \usepackage{textcomp} % defines textquotesingle
    % Hack from http://tex.stackexchange.com/a/47451/13684:
    \AtBeginDocument{%
        \def\PYZsq{\textquotesingle}% Upright quotes in Pygmentized code
    }
    \usepackage{upquote} % Upright quotes for verbatim code
    \usepackage{eurosym} % defines \euro
    \usepackage[mathletters]{ucs} % Extended unicode (utf-8) support
    \usepackage[utf8x]{inputenc} % Allow utf-8 characters in the tex document
    \usepackage{fancyvrb} % verbatim replacement that allows latex
    \usepackage{grffile} % extends the file name processing of package graphics 
                         % to support a larger range 
    % The hyperref package gives us a pdf with properly built
    % internal navigation ('pdf bookmarks' for the table of contents,
    % internal cross-reference links, web links for URLs, etc.)
    \usepackage{hyperref}
    \usepackage{longtable} % longtable support required by pandoc >1.10
    \usepackage{booktabs}  % table support for pandoc > 1.12.2
    \usepackage[inline]{enumitem} % IRkernel/repr support (it uses the enumerate* environment)
    \usepackage[normalem]{ulem} % ulem is needed to support strikethroughs (\sout)
                                % normalem makes italics be italics, not underlines
    

    
    
    % Colors for the hyperref package
    \definecolor{urlcolor}{rgb}{0,.145,.698}
    \definecolor{linkcolor}{rgb}{.71,0.21,0.01}
    \definecolor{citecolor}{rgb}{.12,.54,.11}

    % ANSI colors
    \definecolor{ansi-black}{HTML}{3E424D}
    \definecolor{ansi-black-intense}{HTML}{282C36}
    \definecolor{ansi-red}{HTML}{E75C58}
    \definecolor{ansi-red-intense}{HTML}{B22B31}
    \definecolor{ansi-green}{HTML}{00A250}
    \definecolor{ansi-green-intense}{HTML}{007427}
    \definecolor{ansi-yellow}{HTML}{DDB62B}
    \definecolor{ansi-yellow-intense}{HTML}{B27D12}
    \definecolor{ansi-blue}{HTML}{208FFB}
    \definecolor{ansi-blue-intense}{HTML}{0065CA}
    \definecolor{ansi-magenta}{HTML}{D160C4}
    \definecolor{ansi-magenta-intense}{HTML}{A03196}
    \definecolor{ansi-cyan}{HTML}{60C6C8}
    \definecolor{ansi-cyan-intense}{HTML}{258F8F}
    \definecolor{ansi-white}{HTML}{C5C1B4}
    \definecolor{ansi-white-intense}{HTML}{A1A6B2}

    % commands and environments needed by pandoc snippets
    % extracted from the output of `pandoc -s`
    \providecommand{\tightlist}{%
      \setlength{\itemsep}{0pt}\setlength{\parskip}{0pt}}
    \DefineVerbatimEnvironment{Highlighting}{Verbatim}{commandchars=\\\{\}}
    % Add ',fontsize=\small' for more characters per line
    \newenvironment{Shaded}{}{}
    \newcommand{\KeywordTok}[1]{\textcolor[rgb]{0.00,0.44,0.13}{\textbf{{#1}}}}
    \newcommand{\DataTypeTok}[1]{\textcolor[rgb]{0.56,0.13,0.00}{{#1}}}
    \newcommand{\DecValTok}[1]{\textcolor[rgb]{0.25,0.63,0.44}{{#1}}}
    \newcommand{\BaseNTok}[1]{\textcolor[rgb]{0.25,0.63,0.44}{{#1}}}
    \newcommand{\FloatTok}[1]{\textcolor[rgb]{0.25,0.63,0.44}{{#1}}}
    \newcommand{\CharTok}[1]{\textcolor[rgb]{0.25,0.44,0.63}{{#1}}}
    \newcommand{\StringTok}[1]{\textcolor[rgb]{0.25,0.44,0.63}{{#1}}}
    \newcommand{\CommentTok}[1]{\textcolor[rgb]{0.38,0.63,0.69}{\textit{{#1}}}}
    \newcommand{\OtherTok}[1]{\textcolor[rgb]{0.00,0.44,0.13}{{#1}}}
    \newcommand{\AlertTok}[1]{\textcolor[rgb]{1.00,0.00,0.00}{\textbf{{#1}}}}
    \newcommand{\FunctionTok}[1]{\textcolor[rgb]{0.02,0.16,0.49}{{#1}}}
    \newcommand{\RegionMarkerTok}[1]{{#1}}
    \newcommand{\ErrorTok}[1]{\textcolor[rgb]{1.00,0.00,0.00}{\textbf{{#1}}}}
    \newcommand{\NormalTok}[1]{{#1}}
    
    % Additional commands for more recent versions of Pandoc
    \newcommand{\ConstantTok}[1]{\textcolor[rgb]{0.53,0.00,0.00}{{#1}}}
    \newcommand{\SpecialCharTok}[1]{\textcolor[rgb]{0.25,0.44,0.63}{{#1}}}
    \newcommand{\VerbatimStringTok}[1]{\textcolor[rgb]{0.25,0.44,0.63}{{#1}}}
    \newcommand{\SpecialStringTok}[1]{\textcolor[rgb]{0.73,0.40,0.53}{{#1}}}
    \newcommand{\ImportTok}[1]{{#1}}
    \newcommand{\DocumentationTok}[1]{\textcolor[rgb]{0.73,0.13,0.13}{\textit{{#1}}}}
    \newcommand{\AnnotationTok}[1]{\textcolor[rgb]{0.38,0.63,0.69}{\textbf{\textit{{#1}}}}}
    \newcommand{\CommentVarTok}[1]{\textcolor[rgb]{0.38,0.63,0.69}{\textbf{\textit{{#1}}}}}
    \newcommand{\VariableTok}[1]{\textcolor[rgb]{0.10,0.09,0.49}{{#1}}}
    \newcommand{\ControlFlowTok}[1]{\textcolor[rgb]{0.00,0.44,0.13}{\textbf{{#1}}}}
    \newcommand{\OperatorTok}[1]{\textcolor[rgb]{0.40,0.40,0.40}{{#1}}}
    \newcommand{\BuiltInTok}[1]{{#1}}
    \newcommand{\ExtensionTok}[1]{{#1}}
    \newcommand{\PreprocessorTok}[1]{\textcolor[rgb]{0.74,0.48,0.00}{{#1}}}
    \newcommand{\AttributeTok}[1]{\textcolor[rgb]{0.49,0.56,0.16}{{#1}}}
    \newcommand{\InformationTok}[1]{\textcolor[rgb]{0.38,0.63,0.69}{\textbf{\textit{{#1}}}}}
    \newcommand{\WarningTok}[1]{\textcolor[rgb]{0.38,0.63,0.69}{\textbf{\textit{{#1}}}}}
    
    
    % Define a nice break command that doesn't care if a line doesn't already
    % exist.
    \def\br{\hspace*{\fill} \\* }
    % Math Jax compatability definitions
    \def\gt{>}
    \def\lt{<}
    % Document parameters
    \title{00 Initial Pinlte Testing postmortem }
    
    
    

    % Pygments definitions
    
\makeatletter
\def\PY@reset{\let\PY@it=\relax \let\PY@bf=\relax%
    \let\PY@ul=\relax \let\PY@tc=\relax%
    \let\PY@bc=\relax \let\PY@ff=\relax}
\def\PY@tok#1{\csname PY@tok@#1\endcsname}
\def\PY@toks#1+{\ifx\relax#1\empty\else%
    \PY@tok{#1}\expandafter\PY@toks\fi}
\def\PY@do#1{\PY@bc{\PY@tc{\PY@ul{%
    \PY@it{\PY@bf{\PY@ff{#1}}}}}}}
\def\PY#1#2{\PY@reset\PY@toks#1+\relax+\PY@do{#2}}

\expandafter\def\csname PY@tok@w\endcsname{\def\PY@tc##1{\textcolor[rgb]{0.73,0.73,0.73}{##1}}}
\expandafter\def\csname PY@tok@c\endcsname{\let\PY@it=\textit\def\PY@tc##1{\textcolor[rgb]{0.25,0.50,0.50}{##1}}}
\expandafter\def\csname PY@tok@cp\endcsname{\def\PY@tc##1{\textcolor[rgb]{0.74,0.48,0.00}{##1}}}
\expandafter\def\csname PY@tok@k\endcsname{\let\PY@bf=\textbf\def\PY@tc##1{\textcolor[rgb]{0.00,0.50,0.00}{##1}}}
\expandafter\def\csname PY@tok@kp\endcsname{\def\PY@tc##1{\textcolor[rgb]{0.00,0.50,0.00}{##1}}}
\expandafter\def\csname PY@tok@kt\endcsname{\def\PY@tc##1{\textcolor[rgb]{0.69,0.00,0.25}{##1}}}
\expandafter\def\csname PY@tok@o\endcsname{\def\PY@tc##1{\textcolor[rgb]{0.40,0.40,0.40}{##1}}}
\expandafter\def\csname PY@tok@ow\endcsname{\let\PY@bf=\textbf\def\PY@tc##1{\textcolor[rgb]{0.67,0.13,1.00}{##1}}}
\expandafter\def\csname PY@tok@nb\endcsname{\def\PY@tc##1{\textcolor[rgb]{0.00,0.50,0.00}{##1}}}
\expandafter\def\csname PY@tok@nf\endcsname{\def\PY@tc##1{\textcolor[rgb]{0.00,0.00,1.00}{##1}}}
\expandafter\def\csname PY@tok@nc\endcsname{\let\PY@bf=\textbf\def\PY@tc##1{\textcolor[rgb]{0.00,0.00,1.00}{##1}}}
\expandafter\def\csname PY@tok@nn\endcsname{\let\PY@bf=\textbf\def\PY@tc##1{\textcolor[rgb]{0.00,0.00,1.00}{##1}}}
\expandafter\def\csname PY@tok@ne\endcsname{\let\PY@bf=\textbf\def\PY@tc##1{\textcolor[rgb]{0.82,0.25,0.23}{##1}}}
\expandafter\def\csname PY@tok@nv\endcsname{\def\PY@tc##1{\textcolor[rgb]{0.10,0.09,0.49}{##1}}}
\expandafter\def\csname PY@tok@no\endcsname{\def\PY@tc##1{\textcolor[rgb]{0.53,0.00,0.00}{##1}}}
\expandafter\def\csname PY@tok@nl\endcsname{\def\PY@tc##1{\textcolor[rgb]{0.63,0.63,0.00}{##1}}}
\expandafter\def\csname PY@tok@ni\endcsname{\let\PY@bf=\textbf\def\PY@tc##1{\textcolor[rgb]{0.60,0.60,0.60}{##1}}}
\expandafter\def\csname PY@tok@na\endcsname{\def\PY@tc##1{\textcolor[rgb]{0.49,0.56,0.16}{##1}}}
\expandafter\def\csname PY@tok@nt\endcsname{\let\PY@bf=\textbf\def\PY@tc##1{\textcolor[rgb]{0.00,0.50,0.00}{##1}}}
\expandafter\def\csname PY@tok@nd\endcsname{\def\PY@tc##1{\textcolor[rgb]{0.67,0.13,1.00}{##1}}}
\expandafter\def\csname PY@tok@s\endcsname{\def\PY@tc##1{\textcolor[rgb]{0.73,0.13,0.13}{##1}}}
\expandafter\def\csname PY@tok@sd\endcsname{\let\PY@it=\textit\def\PY@tc##1{\textcolor[rgb]{0.73,0.13,0.13}{##1}}}
\expandafter\def\csname PY@tok@si\endcsname{\let\PY@bf=\textbf\def\PY@tc##1{\textcolor[rgb]{0.73,0.40,0.53}{##1}}}
\expandafter\def\csname PY@tok@se\endcsname{\let\PY@bf=\textbf\def\PY@tc##1{\textcolor[rgb]{0.73,0.40,0.13}{##1}}}
\expandafter\def\csname PY@tok@sr\endcsname{\def\PY@tc##1{\textcolor[rgb]{0.73,0.40,0.53}{##1}}}
\expandafter\def\csname PY@tok@ss\endcsname{\def\PY@tc##1{\textcolor[rgb]{0.10,0.09,0.49}{##1}}}
\expandafter\def\csname PY@tok@sx\endcsname{\def\PY@tc##1{\textcolor[rgb]{0.00,0.50,0.00}{##1}}}
\expandafter\def\csname PY@tok@m\endcsname{\def\PY@tc##1{\textcolor[rgb]{0.40,0.40,0.40}{##1}}}
\expandafter\def\csname PY@tok@gh\endcsname{\let\PY@bf=\textbf\def\PY@tc##1{\textcolor[rgb]{0.00,0.00,0.50}{##1}}}
\expandafter\def\csname PY@tok@gu\endcsname{\let\PY@bf=\textbf\def\PY@tc##1{\textcolor[rgb]{0.50,0.00,0.50}{##1}}}
\expandafter\def\csname PY@tok@gd\endcsname{\def\PY@tc##1{\textcolor[rgb]{0.63,0.00,0.00}{##1}}}
\expandafter\def\csname PY@tok@gi\endcsname{\def\PY@tc##1{\textcolor[rgb]{0.00,0.63,0.00}{##1}}}
\expandafter\def\csname PY@tok@gr\endcsname{\def\PY@tc##1{\textcolor[rgb]{1.00,0.00,0.00}{##1}}}
\expandafter\def\csname PY@tok@ge\endcsname{\let\PY@it=\textit}
\expandafter\def\csname PY@tok@gs\endcsname{\let\PY@bf=\textbf}
\expandafter\def\csname PY@tok@gp\endcsname{\let\PY@bf=\textbf\def\PY@tc##1{\textcolor[rgb]{0.00,0.00,0.50}{##1}}}
\expandafter\def\csname PY@tok@go\endcsname{\def\PY@tc##1{\textcolor[rgb]{0.53,0.53,0.53}{##1}}}
\expandafter\def\csname PY@tok@gt\endcsname{\def\PY@tc##1{\textcolor[rgb]{0.00,0.27,0.87}{##1}}}
\expandafter\def\csname PY@tok@err\endcsname{\def\PY@bc##1{\setlength{\fboxsep}{0pt}\fcolorbox[rgb]{1.00,0.00,0.00}{1,1,1}{\strut ##1}}}
\expandafter\def\csname PY@tok@kc\endcsname{\let\PY@bf=\textbf\def\PY@tc##1{\textcolor[rgb]{0.00,0.50,0.00}{##1}}}
\expandafter\def\csname PY@tok@kd\endcsname{\let\PY@bf=\textbf\def\PY@tc##1{\textcolor[rgb]{0.00,0.50,0.00}{##1}}}
\expandafter\def\csname PY@tok@kn\endcsname{\let\PY@bf=\textbf\def\PY@tc##1{\textcolor[rgb]{0.00,0.50,0.00}{##1}}}
\expandafter\def\csname PY@tok@kr\endcsname{\let\PY@bf=\textbf\def\PY@tc##1{\textcolor[rgb]{0.00,0.50,0.00}{##1}}}
\expandafter\def\csname PY@tok@bp\endcsname{\def\PY@tc##1{\textcolor[rgb]{0.00,0.50,0.00}{##1}}}
\expandafter\def\csname PY@tok@fm\endcsname{\def\PY@tc##1{\textcolor[rgb]{0.00,0.00,1.00}{##1}}}
\expandafter\def\csname PY@tok@vc\endcsname{\def\PY@tc##1{\textcolor[rgb]{0.10,0.09,0.49}{##1}}}
\expandafter\def\csname PY@tok@vg\endcsname{\def\PY@tc##1{\textcolor[rgb]{0.10,0.09,0.49}{##1}}}
\expandafter\def\csname PY@tok@vi\endcsname{\def\PY@tc##1{\textcolor[rgb]{0.10,0.09,0.49}{##1}}}
\expandafter\def\csname PY@tok@vm\endcsname{\def\PY@tc##1{\textcolor[rgb]{0.10,0.09,0.49}{##1}}}
\expandafter\def\csname PY@tok@sa\endcsname{\def\PY@tc##1{\textcolor[rgb]{0.73,0.13,0.13}{##1}}}
\expandafter\def\csname PY@tok@sb\endcsname{\def\PY@tc##1{\textcolor[rgb]{0.73,0.13,0.13}{##1}}}
\expandafter\def\csname PY@tok@sc\endcsname{\def\PY@tc##1{\textcolor[rgb]{0.73,0.13,0.13}{##1}}}
\expandafter\def\csname PY@tok@dl\endcsname{\def\PY@tc##1{\textcolor[rgb]{0.73,0.13,0.13}{##1}}}
\expandafter\def\csname PY@tok@s2\endcsname{\def\PY@tc##1{\textcolor[rgb]{0.73,0.13,0.13}{##1}}}
\expandafter\def\csname PY@tok@sh\endcsname{\def\PY@tc##1{\textcolor[rgb]{0.73,0.13,0.13}{##1}}}
\expandafter\def\csname PY@tok@s1\endcsname{\def\PY@tc##1{\textcolor[rgb]{0.73,0.13,0.13}{##1}}}
\expandafter\def\csname PY@tok@mb\endcsname{\def\PY@tc##1{\textcolor[rgb]{0.40,0.40,0.40}{##1}}}
\expandafter\def\csname PY@tok@mf\endcsname{\def\PY@tc##1{\textcolor[rgb]{0.40,0.40,0.40}{##1}}}
\expandafter\def\csname PY@tok@mh\endcsname{\def\PY@tc##1{\textcolor[rgb]{0.40,0.40,0.40}{##1}}}
\expandafter\def\csname PY@tok@mi\endcsname{\def\PY@tc##1{\textcolor[rgb]{0.40,0.40,0.40}{##1}}}
\expandafter\def\csname PY@tok@il\endcsname{\def\PY@tc##1{\textcolor[rgb]{0.40,0.40,0.40}{##1}}}
\expandafter\def\csname PY@tok@mo\endcsname{\def\PY@tc##1{\textcolor[rgb]{0.40,0.40,0.40}{##1}}}
\expandafter\def\csname PY@tok@ch\endcsname{\let\PY@it=\textit\def\PY@tc##1{\textcolor[rgb]{0.25,0.50,0.50}{##1}}}
\expandafter\def\csname PY@tok@cm\endcsname{\let\PY@it=\textit\def\PY@tc##1{\textcolor[rgb]{0.25,0.50,0.50}{##1}}}
\expandafter\def\csname PY@tok@cpf\endcsname{\let\PY@it=\textit\def\PY@tc##1{\textcolor[rgb]{0.25,0.50,0.50}{##1}}}
\expandafter\def\csname PY@tok@c1\endcsname{\let\PY@it=\textit\def\PY@tc##1{\textcolor[rgb]{0.25,0.50,0.50}{##1}}}
\expandafter\def\csname PY@tok@cs\endcsname{\let\PY@it=\textit\def\PY@tc##1{\textcolor[rgb]{0.25,0.50,0.50}{##1}}}

\def\PYZbs{\char`\\}
\def\PYZus{\char`\_}
\def\PYZob{\char`\{}
\def\PYZcb{\char`\}}
\def\PYZca{\char`\^}
\def\PYZam{\char`\&}
\def\PYZlt{\char`\<}
\def\PYZgt{\char`\>}
\def\PYZsh{\char`\#}
\def\PYZpc{\char`\%}
\def\PYZdl{\char`\$}
\def\PYZhy{\char`\-}
\def\PYZsq{\char`\'}
\def\PYZdq{\char`\"}
\def\PYZti{\char`\~}
% for compatibility with earlier versions
\def\PYZat{@}
\def\PYZlb{[}
\def\PYZrb{]}
\makeatother


    % Exact colors from NB
    \definecolor{incolor}{rgb}{0.0, 0.0, 0.5}
    \definecolor{outcolor}{rgb}{0.545, 0.0, 0.0}



    
    % Prevent overflowing lines due to hard-to-break entities
    \sloppy 
    % Setup hyperref package
    \hypersetup{
      breaklinks=true,  % so long urls are correctly broken across lines
      colorlinks=true,
      urlcolor=urlcolor,
      linkcolor=linkcolor,
      citecolor=citecolor,
      }
    % Slightly bigger margins than the latex defaults
    
    \geometry{verbose,tmargin=1in,bmargin=1in,lmargin=1in,rmargin=1in}
    
    

    \begin{document}
    
    
    \maketitle
    
    

    
    \section{Pintle V1 Test for 2.2kN Liquid Propellant Rocket
Engine}\label{pintle-v1-test-for-2.2kn-liquid-propellant-rocket-engine}

Dates of testing: 1/19/19, 1/26/19 \& 2/3/19

    \section{Introduction}\label{introduction}

The Portland State Aerospace Society is in the process of testing its
\(2.2kN\) (\(500 lb_f\)) regeneratively cooled bi-propellant rocket
motor. One of the critical steps prior to hot firing the motor is the
testing and validating of the propellant injector. For this motor, a
pintle style injector was selected and an aluminum test article was
machined according to the designs provided by the 2015 Capstone team.

    \section{Pintle Design:}\label{pintle-design}

The engine was designed to employ a liquid oxygen (LOX) centered pintle
design. In this design, LOX flows throu the center of the pintle and is
sprayed outward in a radial pattern. The fuel enters a chamber
surrounding the base of the pintle and enters the engine through the
annular gap between the outer radius of the pintle and the engine. (See
Figures 1 \& 2) Inside of the combustion chamber, the two propellants
impinge upon eachother at right angles. This, ideally, results in a cone
shaped spray of well mixed fuel and oxidizer.

\begin{quote}
Figure 1: Cross section of 2.2 kN engine. Fuel flow path shown in red,
LOX flow path shown in green.
\end{quote}

\begin{quote}
Figure 2: Cross section of Pintle with basic dimensions for flow paths.
\end{quote}

\subsection{Relevant Design
Requirements:}\label{relevant-design-requirements}

The pintle injector needs to meet the following requirements:

\begin{enumerate}
\def\labelenumi{\arabic{enumi}.}
\tightlist
\item
  Oxidizer: Liquid Oxygen (LOX)
\item
  Fuel: 70\% Isoproply Alchohol (IPA), 30\% water
\item
  Meet mass flow rates at a pressure differential between 15-20 \% of
  chamber pressure:

  \begin{itemize}
  \tightlist
  \item
    Engine chamber pressure 350 psi (as designed)
  \item
    Target pressure differential: 53-70 psi\\
  \end{itemize}
\item
  LOX mass flow rate of 0.9 lbm/s (0.4082 kg/s)
\item
  Fuel mass flow rate of 1.16 lbm/s (0.5262 kg/s)
\end{enumerate}

    \section{Objectives}\label{objectives}

Desired data and outcomes:

\begin{itemize}
\tightlist
\item
  Determine loss coefficients (\(C_L\))

  \begin{itemize}
  \tightlist
  \item
    Pintle (LOX path)
  \item
    Annulus (Fuel path)
  \end{itemize}
\item
  Verify desired flow rates acheived at a delta pressure of 70 psi
\end{itemize}

    \section{Equations and Theory}\label{equations-and-theory}

The first testing objective was to determine the loss coefficients for
the Pintle and Annulus flow paths. For practical, cost, and safety
reasons the initial testing was performed using water.

Flow coefficient for the style of orifice plate used in testing {[}1{]};

\begin{equation}
C_v = \frac{Q_{gpm}}{\sqrt{\Delta P}}
\label{eq1}
\tag{1}
\end{equation}

Where;

\$C\_v = \$ the flow coefficient

\(Q_{gpm} =\) is the volumentric flow rate in gallons per minute

\$\Delta P = \$ is the difference in pressure across the orifice

\$ \textbackslash{} \$

Mass flow rate:

\begin{equation}
 \dot{m} = \rho VA
\label{eq2}
\tag{2}
\end{equation}

Energy Equation, with elevation, velocity, and head loss due to friction
neglected:

\begin{equation}
P_1  = P_2 + C_L\frac{\rho V_2^2}{2}
\label{eq3}
\tag{3}
\end{equation}

Combining equation \ref{eq2} and \ref{eq3} by substituting
\(V = \frac{\dot{m}}{\rho A}\) and rearranging gives

\begin{equation}
\dot{m} = C_LA \sqrt{2\rho(\Delta P)}
\label{eq4}
\tag{4}
\end{equation}

Where;

\$C\_L = \$ Loss coefficients for Pintle and Annulus: (to be calculated)

\(A =\) Flow area(s) of Pintle/Annulus

\(\dot{m} =\) Mass Flow Rate (from flow meter)

\(\Delta P =\) Pressure upstream of Pintle/Annulus (gauge) - atmospheric
pressure (0 psi)

\$\rho = \$ Density of water (calculate based on temperature)

Note: Each flow path has multiple changes to flow area between point of
measurement and the atmosphere. As a result, it is being assumed
(possibly incorrectly) that the \(C_L\) and discharge area \(A\) are
coupled and the loss coefficient should incorporate both with the
understanding that the coefficient is only good for this specific
version of the pintle design;

\begin{equation}
C_LA = C_A = \frac{\dot{m}}{\sqrt{2\rho(\Delta P)}}
\label{eq5}
\tag{5}
\end{equation}

From this it was determined that in order to solve for \(C_A\) the
experimental set up needed to determine the mass flow rate and pressure
drop.

    \section{Methods}\label{methods}

The following test set up was designed to obtain the mass flow and
pressure measurements.

\begin{quote}
Figure 3: Schematic of Pintle Test Layout
\end{quote}

\subsection{Calibration:}\label{calibration}

\textbf{Pressure Transducers:}

The 5 pressure transducers were calibrated simultaneously on the stand
using compressed air for measurements at 0, 25, 50, 75, and 100 PSIG.
The system pressure was set using a bourdon tube style pressure gauge.
This calibration was repeated at the start of each day of testing.

\textbf{Flow measuremnt:}

Our initial test set up included a turbine flow meter (not shown in
schematic) located between the manual linear globe valve and pressure
transducer 2. This flow meter was found to be malfunctioning. We decided
to move forward with the test using only the calibrated orifice plate to
determine the total mass flow rate for the pintle and annulus.\\
The orifice was calibrated using multiple input pressures. The flow rate
was determined using a stop watch and a graduated container. Timing for
each of the tests began at the \(1/2\) gallon mark and ended at the 3.5
gallon mark in order allow the flow to reach a steady flow condition.
Times were recorded every quarter gallon and then plotted to determine
flow rates. Each of the flow rates was then plotted against the
corresponding pressure drop and compared to the average \(C_v\)
calculated from the experiment.

\begin{quote}
Figure 4: Flow rate curve and experimental flow rate data for calibrated
orifice. Experimental data indicated that the accuracy of the orifice
drops off steeply at pressure differences over 30 psi. More on this in
the Issues section
\end{quote}

After determining the \(C_v\) of the main orifice, the pintle test
article was attached to the system and the following test proceedure was
followed:

    \section{Test procedure.}\label{test-procedure.}

\begin{enumerate}
\def\labelenumi{\arabic{enumi}.}
\item
  With valve to fuel circuit closed and valve to LOX circuit open. Test
  LOX circuit at tank pressures of 50, 100, and 200 psi\(^\dagger\). Use
  flow rate data from orifice plate and pressure recorded above the
  pintle to determine loss coefficient for pintle flow path,
  \(C_{A,pintle}\).
\item
  With valve to LOX circuit closed and valve to Fuel circuit open. Test
  fuel circuit at tank pressures of 50, 100, and 200 psi\(^\dagger\).
  Use flow rate data from orifice plate and pressure recorded above the
  annulus to determine loss coefficient for annulus flow path,
  \(C_{A,annulus}\).
\item
  Starting with Fuel circuit valve open, and LOX circuit valve closed,
  adjust flow through pintle using LOX circuit glove valve unitl a
  45\(^{\circ}\) spray is observed. Spray angle to be measured visually
  using a large laser cut protractor with radial spokes every
  15\(^{\circ}\).
\item
  With spray angle adjusted to 45\(^{\circ}\), record total flow rate
  using orifice plate. Determine LOX circuit and Fuel Circuit flow rates
  using previously calculated loss coefficients, \(C_{A,pintle}\) and
  \(C_{A,annulus}\) to determine the flow rates for each fluid circuit.
  Double
\item
  Repeat steps 3 and 4 at 50, 100, and 200 psi.
\item
  After testing, check flow rates for each circuit against the
  continuity equation: \[Q_{total} = Q_{pintle}+Q_{annulus} \]
\end{enumerate}

\$ \textbackslash{} \$

\begin{quote}
\(\dagger\) Tank pressures used for each day of testing. - 1/19: 50,
100, and 200 PSI, Nitrogen was used for the 200 PSI test. - 1/26: 25,
50, 75, 100 PSI - 2/3: 100 PSI only
\end{quote}

    \section{Known Issues with experimental set
up}\label{known-issues-with-experimental-set-up}

A number of issues were encountered during testing which potentially
invalidate some of the collected data. These are being written up as
lessons learned for future testing.

\subsection{Damaged turbine flow
meter}\label{damaged-turbine-flow-meter}

When inspected, it was determined that debris had gotten into the flow
meter and damaged the turbine. Root cause is believed to be mis-use due
to lack of adequate documentation on proper use, maintainence, and
storage the flow meter. The turbine flow meter was located just
downstream of the globe valve for the tests performed on 1/19 and was
removed prior to the the tests performed on 1/26.\\
- Recommendation for when a new flow meter is purchased, or otherwise
obtained; both the flow meter and a summery write up of how to use,
calibrate, and maintain the flow meter be bagged in a labeled container.
Any team which desires to use the flow meter must first review the
instructions and is responsible for following them.

\subsection{Orifice flow meter set up}\label{orifice-flow-meter-set-up}

The first issue with our orifice plate set up was that we had difficulty
locating an orfice plate which could measure the flow rates we require
in a 0.5 inch tube. The closest found was from
\href{http://catalog.okeefecontrols.com/item/precision-metal-orifices-npt-connections/precision-metal-orifices-npt-adapter/h-125-br}{O'Keefe
Controls} with a flow rate of 3.096 gpm of water at 70 psi. This orific
was modified based on sizing calculations provided by the manufaturer
with the intention to test the orifice across the expected flow ranges
to determine the correct orifice \(C_v\).

The second, larger, issue is in where the pressure tranducers were
located relative to the orifice. The pressure transducers were placed as
close to the orifice plate as our fittings would allow, rather than at
one of the standard distances called out in various flow measurement
references. At the time the orifice meter was constructed it was
believed that we could compensate for the placement of the pressure
transducers via calibration.

\begin{quote}
Figure 5: Orifice and pressure transducer set up used for experiment.
Downstream pressure transducer (PT \#3) is approximately 5.5 tube
diameters downstream of orifice.
\end{quote}

For the initial calibration tests without the pintle, this appeared to
be sufficient. However, when pintle was attached, we were not able to
detect a pressure drop across the orifice plate when testing the LOX
circuit.

\begin{quote}
Figure 6: Pressure readings for LOX circuit flow only. Pressure drop
across the orifice was not detectable with testing set up used. From the
data, we were not able to determine the flow rate of the water for the
series of tests performed on 1/26/19
\end{quote}

\subsection{Leaky gate valve}\label{leaky-gate-valve}

On the first day of testing it was noted that our manual globe valve was
not water tight. As we did not have a replacement on hand, we proceeded
with the tests with the understanding that there would be some water
flow and pressure loss through the LOX circuit when testing the Fuel
circuit. This valve was replaced prior to the testing performed on 1/26
with a properly functioning globe valve.

\subsection{Loose wire conncetion between DAQ and pressure
transducers}\label{loose-wire-conncetion-between-daq-and-pressure-transducers}

When calibrating our pressure transducers on 1/23, we found a loose wire
which was producing a high level of noise. Once properly secured our
pressure transducer readings became significanly less scattered. When
comparing the collected data between 1/19 and subsequent testing dates
it became obvious that the data collected from 1/19 was affected by
this, or a similar issue. This issue potentially also affected the
original orifice flow rate data (shown in Figure 4 above), which was not
realized until after testing was completed.

\subsection{Debris trapped in the
annulus}\label{debris-trapped-in-the-annulus}

On the first day of testing it was noted that the spray from the annulus
was uneven. When a \(45^{\circ}\) spray test was performed, we observed
poor mixing and atomization. Prior to the testing on 1/26 the pintle
test article was disassembled and fully cleaned. While cleaning some
metal chips were found trapped in the annulus (Figure 5). After this
debris was removed, we observed significant improvement in the mixing
and atomization (figure 6).

\begin{quote}
Figure 5: Debris found in annulus.
\end{quote}

\begin{quote}
Figure 6: Before and after images from spray angle testing, video can be
found
\href{https://github.com/psas/liquid-engine-test-stand/blob/master/Pintle/Pintle_V1\%20testing/Images/side\%20by\%20side\%20spray\%20angle.wmv}{here}.
\end{quote}

"Where accuracy is important, direct flow calibration is recommended.
Water flow calibration, using tap water, a stop watch, and a glass
graduate (or a pail and scale) to measure total flow, is readily carried
out in the instrument shop or laboratory. For viscous liquids,
calibration with the working fluid is preferable, because viscosity has
a substantial effect on most units. Calibration across the working range
is recommended, given that precise conformity to the square law may not
exist." - Instrument Engineers Handbook

    \section{Results}\label{results}

The pintle testing took place over the course of three weekends. A
summary of each days testing and the results are presented here, with
links at the bottom of this section to the raw data collected for each
day of testing.

\subsection{Day 1: 1/19/2019}\label{day-1-1192019}

As stated in the issues section, the calibration done on this date was
adversely affected by loose wiring. While we did obtain data, the the
spread and some of the pressure readings obtained raise questions about
its accuracy.

\href{https://github.com/psas/liquid-engine-test-stand/tree/master/analysis/Pintle_V1\%20testing/Pintle\%20Test\%2001-19-2019}{Raw
test data for 1/19/2019}

\subsection{Day 2: 1/26/2019}\label{day-2-1262019}

Some interesting data was collected on the second day, however due to a
combination of the

As noted in the issues section, the flow rate data from the orifice is
suspect. Unfortunately without knowing how to back out the mass flow
rates from our data, we were unable to determine if the Pintle achieved
the required flow rates.

As an aside, we did note some interesting pressure phenomena during the
testing. Figure 7 shows and example of results we obtained when we
varied the flow through the LOX circuit using the globe valve. At the
start of the test the globe valve was full closed and was fully opened
then closed over approximately 40 seconds. For the data shown, the
upstream tank pressure was set to 75 PSIG.

At approximagely 15 seconds the a hissing sound (consistent with
cavitation) was coming from the globe valve. The hissing ceased at
approximatly 25 seconds into the test.

A substantial pressure rise in the LOX circuit was observed in the 25,
50 and all three of the 75 PSI tests where this test was repeated.

Also noticed was that as the pressure increased in the LOX circuit, the
pressure read by pressure transducer 3 (located just downstream of the
orifice) also increased. This resulted in the differential pressure
recorded across the orifice going down as we increased the flow through
the LOX circuit. The recorded pressure differential became negative
during the same time period as the hissing sound was observed in the
valve.

\begin{quote}
Figure 5: Graph of spray angle test. Tank pressure was set to 100 PSIG
at at test start. The globe valve used to control the flow to LOX
circuit was changed from fully closed, to fully open and back again.
Valve controling flow to fuel circuit remained fully open for entire
test.
\end{quote}

Initial efforts were made to try and understand what caused the pressure
increase in the LOX circuit and several theories have been sugested. At
this time we have not pursued looking verifying these theories. It is
expected that a better testing set up with independant flow paths for
each working fluid will provide more accurate and usable flow data.

When running tests through only the LOX or Fuel circuits, unusual
pressure spikes were not observed, nor did we note any indications of
possible cavitation.

\href{https://github.com/psas/liquid-engine-test-stand/tree/master/analysis/Pintle_V1\%20testing/Pintle\%20Test\%2001-26-2019}{Raw
test data for 1/26/2019}

\subsection{Day 3: 2/3/2019}\label{day-3-232019}

The agenda for day 3 of testing was to test the pintle and annulus at a
pressure within the target range and measure the flow rate.

Only two pressure transducers were calibrated for these tests, one for
each flow path.

To measure the flow rate we altered our set up slightly so that the
water leaving the pintle/annulus could be captured in a graduated
bucket. Each of the test runs was videod to enable us to more accurately
determine the flow rates.

During the analysis the calculated flow rates were compared and it was
determined that neither the Pintle nor Annulus met the requirement.
Table 1 shows the average pressure measurements and flow rates along
with the test requirement.

\begin{quote}
Table 1: Flow rate results for Pintle testing. Tests show that the
current design of the pintle does not obtain the needed flow rates.
\end{quote}

\href{https://github.com/psas/liquid-engine-test-stand/tree/master/analysis/Pintle_V1\%20testing/Pintle\%20Test\%2002-03-2019}{Raw
test data for 2/3/2019}

    \section{Next Steps}\label{next-steps}

The pintle needs to be redesigned so that it acheives the required
propellant mass flow rates. In order to better understand the problem, a
new set of flow calculations were created using the asbuilt geometry for
comparision with the experimental results.

These calculations and the initial design work is taking place in a
separate notebook,
\href{https://github.com/psas/liquid-engine-test-stand/blob/master/Pintle/Pintle_V1\%20testing/01\%20Pintle\%20V1\%20Flow\%20Calculations\%20and\%20Analysis.ipynb}{01
Pintle V1 Flow Calculations and Analysis}.

    References:

{[}1{]} \href{Images/Orifice\%20Technical\%20Considerations.pdf}{O'keefe
Controls Technical Considerations Document}

    


    % Add a bibliography block to the postdoc
    
    
    
    \end{document}
